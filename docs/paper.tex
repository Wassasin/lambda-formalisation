\documentclass[a4paper, 10pt]{amsart}
\usepackage[T1]{fontenc}
\usepackage[utf8]{inputenc}
\usepackage[english]{babel}
\usepackage[a4paper]{geometry}

\usepackage{graphicx}
\usepackage{caption}
\usepackage{subcaption}

\usepackage{amsmath}
\usepackage{amssymb}
\usepackage{mathrsfs}
\usepackage{nicefrac}

\usepackage{enumerate}
\usepackage{hyperref}
\usepackage{comment}
\usepackage{float}
\usepackage{parskip}

\usepackage{xspace}
\usepackage{url}

\usepackage{listings}
\usepackage{prettylistings}

\makeatletter
\providecommand*{\twoheadrightarrowfill@}{%
  \arrowfill@\relbar\relbar\twoheadrightarrow
}
\providecommand*{\xtwoheadrightarrow}[2][]{%
  \ext@arrow 0579\twoheadrightarrowfill@{#1}{#2}%
}
\makeatother

\usepackage{prooftree}
\newcommand{\notjustify}{\thickness0em\justifies}
\usepackage[figuresleft]{rotating}

\title{Type checking $\lambda^\rightarrow$ in Coq}
\author{Wouter Geraedts}

\begin{document}

	\maketitle

	I wrote a type checker for $\lambda^\rightarrow$ in Coq:

	\begin{lstlisting}
	Fixpoint type_check (g : context) (T : term) {struct T}
	: {A : type | has_type g T A} + {forall A : type, ~has_type g T A}
	\end{lstlisting}

	Meaning that aside from checking whether a term has a given type, also prove that otherwise the term has no applicable type.
	Compared to only checking the first property, this required an additional ~200 lines of proof.

	\section{Definitions}

	Variables are strings.
	\begin{lstlisting}
	Definition var := string.
	\end{lstlisting}

	Types are as expected:
	\[\mathbf{Type} := \mathbf{Var} ~|~ \mathbf{Type} \rightarrow \mathbf{Type}\]
	\begin{lstlisting}
	Inductive type : Set :=
	| var_type : var -> type
	| fun_type : type -> type -> type.
	\end{lstlisting}

	For terms we re-use the variables.
	I could've also chosen to use a different domain for term-variables.
	Might have been nicer.
	\[\mathbf{Term} := \mathbf{Var} ~|~ \lambda \mathbf{Var} : \mathbf{Type} . \mathbf{Term} ~|~ \mathbf{Term}~\mathbf{Term}\]

	\begin{lstlisting}
	Inductive term : Set :=
	| var_term : var -> term
	| abs_term : (var * type) -> term -> term
	| app_term : term -> term -> term.
	\end{lstlisting}

	Are type context is a list of pairs of variables and types.
	\begin{lstlisting}
	Definition context := list (var * type).
	\end{lstlisting}

	Central to the context, is the existential proof whether or not a variable is contained within the context.
	For this we use the inductive predicate \emph{assoc}.
	\begin{lstlisting}
	Inductive assoc {A B : Set} (a : A) (b : B) : list (A * B) -> Prop :=
	| assoc_base l : assoc a b ((a, b) :: l)
	| assoc_step a0 b0 l : a <> a0 -> assoc a b l -> assoc a b ((a0, b0) :: l).
	\end{lstlisting}

	Finally, the core of our type checker, is the proof that a term has a specific type.
	For this we use the inductive predicate \emph{has\_type}.
	\begin{lstlisting}
	Inductive has_type (g : context) : term -> type -> Prop :=
	| var_has_type : forall v : var, forall A : type,
	  assoc v A g -> has_type g (var_term v) A
	| abs_has_type : forall v : var, forall M : term, forall A B : type,
	  has_type ((v, A) :: g) M B -> has_type g (abs_term (v, A) M) (fun_type A B)
	| app_has_type : forall M N : term, forall A B : type,
	  has_type g M (fun_type A B) -> has_type g N A -> has_type g (app_term M N) B.
	\end{lstlisting}

	\section{Structure of proof}
	
	\begin{figure}[H]
		\includegraphics[width=\textwidth]{graph.pdf}
		\caption{Lemma-wise dependency graph of proof, revealing structure.}
	\end{figure}
	
	General scheme for implementing: when refining, if require inversion, put that part in a lemma.
	As this typically was required when proving anything about the failure-cases, I had to implement lemma's for each of these.
	Note that if I would've sticked to the original assignment, these lemma's would have been superfluous, and could've been replaced by "return Nothing" or similar.

	\section{Considerations}

	Initially, I implemented \emph{assoc} and \emph{has\_type} as bare propositions, instead of inductive predicates.
	Later I saw that writing proofs for these predicates is easier.
	Thus I rewrote the entire proof.

\end{document}
